% !TEX encoding = UTF-8 Unicode

\documentclass{article}
\usepackage[french]{babel}
\author{Groupe 13}
\title{Compte-rendus SAÉ24}
% \date{9 Juin 2022}
\usepackage[margin=1in]{geometry}
\usepackage{subcaption}
\usepackage{listings}
\usepackage{minted}
\usepackage{graphicx}
\usepackage[T1]{fontenc}
\usepackage[colorlinks=true,linkcolor=black,anchorcolor=black,citecolor=black,filecolor=black,menucolor=black,runcolor=black,urlcolor=black]{hyperref}
%\setcounter{tocdepth}{1} % pour la profondeur de la ToC

\usepackage{fancyhdr}
\pagestyle{fancy}
\fancyhf{}
\renewcommand{\headrulewidth}{0pt}
\rfoot{\thepage}
\lfoot{SAÉ24}

\renewcommand{\listoflistingscaption}{Table des codes}
\renewcommand{\listingscaption}{Code}

\begin{document}

\maketitle
\tableofcontents
\listoffigures
\listoflistings

\newpage
\section{Création des VLAN et routage inter-VLAN}
\subsection{VLANs}
Pour commencer nous avons dû créer quatre VLAN sur notre switch ainsi que de mettre en place le routage inter-VLAN. 
Nous avons donc d'abord créé ces VLAN avec les commandes ci-dessous.
\begin{listing}[H]
    \begin{minted}[breaklines]{text}
Switch(config)#int range fastEthernet 0/1-4
Switch(config-if-range)#sw mode access 
Switch(config-if-range)#sw access vlan 10
% Access VLAN does not exist. Creating vlan 10
    \end{minted}
    \caption{Création d'un VLAN}
    \label{reseau:switch:vlans}
\end{listing}
Nous avons répété les commandes en code \ref{reseau:switch:vlans} quatre fois en utilisant quatre interfaces par VLAN ainsi que les numéros 10, 20, 30 et 40. 
Nous avons ensuite donné des noms à ces VLAN avec les commandes \verb|vlan <no>| puis \verb|name <nom>| en mode configuration.

\begin{listing}[H]
    \begin{minted}[breaklines]{text}
VLAN Name                             Status    Ports
---- -------------------------------- --------- -------------------------------
1    default                          active    Fa0/17, Fa0/18, Fa0/19, Fa0/20
                                                Fa0/21, Fa0/22, Fa0/23, Fa0/24
                                                Gig0/1, Gig0/2
10   voix                             active    Fa0/1, Fa0/2, Fa0/3, Fa0/4
20   users                            active    Fa0/5, Fa0/6, Fa0/7, Fa0/8
30   server                           active    Fa0/9, Fa0/10, Fa0/11, Fa0/12
40   admin                            active    Fa0/13, Fa0/14, Fa0/15, Fa0/16
1002 fddi-default                     active    
1003 token-ring-default               active    
1004 fddinet-default                  active    
1005 trnet-default                    active    
    \end{minted}
    \caption{Résultats de "sh vlan brief"}
    \label{reseau:switch:sh-vlan}
\end{listing}

\subsection{Routage inter-VLAN}
Une fois les VLAN correctement crées, nous avons besoin de configurer le routage inter-VLAN en utilisant l'encapsulation dot1Q.
Pour cela, sur notre switch, nous avons choisi le port \verb|Fa0/24| comme port trunk.
\begin{listing}[H]
    \begin{minted}[breaklines]{text}
Switch(config)#int fastEthernet 0/24
Switch(config-if)#sw mode trunk 
Switch(config-if)#sw trunk allowed vlan 10,20,30,40
    \end{minted}
    \caption{Configuration du port trunk}
    \label{reseau:switch:trunk}
\end{listing}
\end{document}